\documentclass{homework}
\usepackage{amsmath}

\author{Omar Reda H Fatani - 2338450}
\class{NE 304: Dr. Mohammed Damoom}
\date{\today}
\title{Homework 1}
\address{Bayt El-Hikmah}

\graphicspath{{./media/}}

\begin{document} \maketitle

\question \\
The $\beta^-$-emitter \textsuperscript{28}Al (half-life 2.30 min) can be produced 
by the radiative capture of neutrons by \textsuperscript{27}Al. The 0.0253-eV cross-section for this reaction is 0.23 b. Suppose 
that a small, 0.01-g aluminum target is placed in a beam of 0.0253-eV neutrons, $\phi = 3 \times 10^8$ neutrons/cm\textsuperscript{2}-sec, 
which strikes the entire target. Calculate:
\begin{enumerate}
    \item[(a)] the neutron density in the beam;
    \item[(b)] the rate at which \textsuperscript{28}Al is produced;
    \item[(c)] the maximum activity (in curies) that can be produced in this experiment. 
\end{enumerate}

\begin{sol}
  \begin{itemize}
      \item{(a)}
      \begin{gather*}
      \text{For thermal neutrons, the velocity is approximately 2200 m/s.} \\
      n = I/n 
        \\
      n = \frac{3\times 10^8}{2200\times10^2} = 1363\, neutron/cm^3
      \end{gather*}
    \item{(b)}
  \begin{gather*}
    \text{From table II.3, } \rho_{\text{Al}} = 2.699 \, \text{g/cm}^3
    \text{ and } N_{\text{Al}} = 0.06024 \times 10^{24} \, \text{Atoms/cm}^3 \\
    \dot{R} = \sigma \phi NAX = 0.23\times 3\times 10^8\times 0.06024 \times 
    \frac{0.01}{2.699} = 15400\,Interactions/sec
  \end{gather*}
    \item{(b)}
  \begin{gather*}
  \text{At saturation activity, production rate = decary rate, } A_{\text{max}} = \dot{R}\\
  A_{\text{max}} = \frac{15400\,Bq}{3.7\times 10^10} = 4.16 \times 10^{-7}\, Ci
  \end{gather*}

  \end{itemize}
 % the empty line bellow is needed to avoid extra space.
  
\end{sol}

\question \\
Calculate the mean free path of 1-eV neutrons in graphite. The total cross-section of
carbon at this energy is 4.8 b.

\begin{sol}
  \begin{gather*}
  \text{From table II.3, } N_{\text{C(graphite)}} = 0.08023\times 10^{-24} \\
  \bar{X} = \frac{1}{\sigma_{\text{t}} N} = \frac{1}{4.8 \times 0.08023} = 2.597\,cm
  \end{gather*}

\end{sol}

\question \\
A beam of 2-MeV neutrons is incident on a slab of heavy water (D\textsubscript{2}O). The total cross-sections of deuterium and oxygen at this energy are 2.6 b and 1.6 b, respectively.

\begin{enumerate}
    \item[(a)] What is the macroscopic total cross-section of D\textsubscript{2}O at 2 MeV?
    \item[(b)] How thick must the slab be to reduce the intensity of the uncollided beam by a factor of 10?
    \item[(c)] If an incident neutron has a collision in the slab, what is the relative probability that it collides with deuterium?
\end{enumerate}

\begin{sol}
  \begin{itemize}
    \item{(a)} 
      \begin{gather*}
      \text{From table II.3, } N_{D_{2}O} = 0.03323\times 10^{-24} \\
      \sigma = 2\sigma_{D} + \sigma_{O} = (2\times 2.6) + (1.6) = 6.8\,b\\
      \textstyle \sum_{t} = \sigma_{\text{t}}N = 6.8 \times 0.03323 = 0.2256\,cm^{-1}
      \end{gather*}
    \item{(b)}
    \begin{gather*}
    \frac{I_{\text{(x)}}}{I_{0}} = e^{-\textstyle \sum_{t}x} \\
    10 = e^{-0.2256x} \text{, } x = 10.21\,cm
    \end{gather*}
    \item{(c)}
    \begin{gather*}
     P(D|collision) = \frac{\textstyle \sum_{D}}{\textstyle \sum_{t}} = \frac{2\sigma_{D}N}{\textstyle \sum_{t}} \\
     P(D|collision) = \frac{2\times 2.6\times 0.03323 }{0.2256} = 0.766 = 76.6\%
    \end{gather*}
    \end{itemize}
  \end{sol}

\question \\
Stainless steel, type 304 having a density of $7.86 \,\text{g/cm}^3$, has been used in some reactors. 
The nominal composition by weight of this material is as follows: carbon, $0.08\%$; chromium, $19\%$; nickel, $10\%$; iron, the remainder. 
Calculate the macroscopic absorption cross-section of SS-304 at $0.0253 \,\text{eV}$.

\begin{sol}
\vspace{1em}
\noindent Since composite materials (alloys) have densities of their own, the atomic densities of each material should be derived in regards to the alloy density.
  \begin{gather*}
    \Sigma_a = \sum_i N_i \sigma_{a,i}
    \qquad \text{with} \qquad
    N_i = \frac{\rho w_i N_A}{A_i} \\
    \text{subsituting them will give: } \\
    \boxed{\Sigma_a = \rho_{alloy} N_A \sum_i \frac{w_i}{A_i}\,\sigma_{a,i}}\\
    \text{Values of }w_{i}, A_{i} \text{ and }\sigma_{a,i}\text{ is obtained from table II.3}\\
    \Sigma_a = (\rho_{alloy})(N_A)[(\frac{w_{C}}{A_{C}}\times \sigma_{C})+(\frac{w_{Cr}}{A_{Cr}}\times \sigma_{Cr})+(\frac{w_{Ni}}{A_{Ni}}\times \sigma_{Ni})+(\frac{w_{Fe}}{A_{Fe}}\times \sigma_{Fe})]\\
    \Sigma_a = (7.86)(0.6022)[(\frac{0.0008}{12}\times 0.0034)+(\frac{0.19}{52}\times 3.1)+(\frac{0.1}{59}\times 4.43)+(\frac{0.7092}{56}\times 2.55)]\\
    \Sigma_a = 0.242\,cm^{-1}
  \end{gather*}
\end{sol}

\question \\
There are no resonances in the total cross-section of $^{12}\text{C}$ 
from 0.01 eV to cover 1 MeV. If the radiative capture cross-section of this nuclide 
at 0.0253 eV is 3.4 mb, what is the value of $\sigma_{\gamma}$ at 1 eV?
\begin{sol}
  \begin{gather*}
  \sigma_{\gamma}(E) = \sigma_{\gamma}(E_{0}) \times \sqrt{\frac{E_{0}}{E}} \\
  \sigma_{\gamma}(1\,eV) = 3.4\times 10^{-27} \times \sqrt{\frac{0.0253}{1}} = 0.541\,mb
  \end{gather*}
\end{sol}

\question \\
The first resonance in the scattering cross-section of the nuclide 
$^{A}Z$ occurs at 1.24 MeV. The separation energies of nuclides 
$^{A-1}Z$, $^{A}Z$, and $^{A+1}Z$ are 7.00, 7.50, and 8.00 MeV, respectively. 
Which nucleus and at what energy above the ground state is the level that 
gives rise to this resonance?

\begin{sol}
  \begin{gather*}
    E^{*} = S_{n}\!\left(^{A+1}Z\right) + E\\
    E^{*} = 8 + 1.24 = 9.24\,MeV
  \end{gather*}
\end{sol}

\question \\
A 2-MeV neutron traveling in water has a head-on collision with an $^{16}\mathrm{O}$ nucleus.
\begin{enumerate}[label={(\alph*)}]
\item What are the energies of the neutron and nucleus after the collision?
\item Would you expect the water molecule involved in the collision to remain intact after the event?
\end{enumerate}

\begin{sol}
  \begin{itemize}
      \item{(a)}
        \begin{gather*}
        \text{Neutron colliding head-on } \rightarrow \theta_{max} = \pi \rightarrow E^{'} = (\frac{A-1}{A+1})^{2}E \\
        E^{'} = (\frac{16-1}{16+1})^{2}\times 2 = 1.557\,MeV\\
        E_{A} = 2 - 1.557 = 0.442\,MeV
    \end{gather*}
       \item{(b)}
      \\
      \\ 
      No, because the oxygen molecule will recoil with an energy of  4 MeV while the H-O bond in the water molecule is held with an energy of 5 eV.
  \end{itemize}
\end{sol}

\question\\
A 1-MeV neutron strikes a $^{12}\mathrm{C}$ nucleus initially at rest. 
If the neutron is elastic scattered through an angle of $90^{\circ}$:
\begin{enumerate}[label=\textbf{(\alph*)}]
\item What is the energy of the scattered neutron?
\item What is the energy of the recoiling nucleus?
\item At what angle does the recoiling nucleus appear?
\end{enumerate}

\begin{sol}
  \begin{itemize}
    \item{(a)}
    \begin{gather*}
      E' = \frac{E}{(A+1)^2} 
      \left[ \cos \vartheta + \sqrt{A^{2} - \sin^{2} \vartheta} \right]^2\\
      E' = \frac{1}{(12+1)^2}[ \cos(90) + \sqrt{12^{2} - \sin^{2}(90)}]^{2} = 0.846\,MeV
    \end{gather*}
    \item{(b)}
    \begin{gather*}
      E_{A} = E - E' = 1 - 0.846 = 0.154 MeV
    \end{gather*}
    \item{(c)}\\\\
    For a 1 MeV neutron moving at 0.46\% speed of light, the lorentz factor is insignificant. Thus, the 
    nonrelativistic model could be applied to derive the angle of recoiled nucleus.
    \img<elastic>[0.5]{Elastic scattering of a neutron by a nucleus.}{Elastic.png}
    \begin{gather*}
      P = P' + P_{A}
      \begin{cases}
        x: \quad P = P'\cos \theta + P_{A} \cos \phi \quad \rightarrow \quad P_{A} \cos \phi = P - P' \cos \theta \\
        y: \quad 0 = P'\sin \theta - P_{A}\sin \phi \quad \rightarrow \quad P_{A} \sin \phi = P' \sin \theta
      \end{cases} \\
      \text{Solving the x and y components by elemination:}\\
      \frac{P_A \sin \phi}{P_A \cos \phi} = \frac{P'\sin \theta}{P - P'\cos \theta}\\
      \text{Given that }P =\sqrt{2mE} \text{, } P' = \sqrt{2mE'} \text{ and }\theta = 90^\circ \text{,} \\
      \tan \phi = \sqrt{\frac{E'}{E}} \rightarrow \phi = \tan^{-1} (\sqrt{\frac{E'}{E}})\\
      \boxed{\phi = \tan^{-1}(\sqrt{\frac{0.846}{1}}) = 42.6^\circ}
    \end{gather*}
  \end{itemize}
\end{sol}

\question\\
Show that the average fractional energy loss in \% in elastic scattering 
for large $A$ is given approximately by:
\begin{gather*}
\frac{\overline{\Delta E}}{E} \simeq \frac{200}{A}
\end{gather*}

\begin{sol}
  \begin{gather*}
  \frac{\overline{\Delta E}}{E} = \frac{1}{2}(1-(\frac{A-1}{A+1}))^{2} \rightarrow f(A) = \frac{1}{2}\left[ \frac{(A + 1)^{2}}{(A + 1)^{2}}-\frac{(A-1)^{2}}{(A+1)^{2}} \right]
  =  \left[ \frac{2A}{(A + 1)^{2}} \right]\\
  \text{Since this is an asymptotic function, with leading term 1/A, set  } c = \lim_{A \to \infty} A f(A) \text{ to find it's coefficient}\\
  \lim_{A \to \infty} A f(A) = \lim_{A \to \infty} \left[ \frac{2A^{2}}{(A + 1)^{2}} \right] \rightarrow \text{Applying L’Hôpital,} \lim_{A \to \infty}\left[ \frac{4A}{2A + 2} \right]
  \rightarrow 2 \\
  \lim_{A \to \infty} A f(A) = 2 \rightarrow  \left[ \lim_{A \to \infty} f(A) \right] \times 100 \% = \left[\frac{2}{A}\right] \times 100\% \\
  \boxed{\frac{\overline{\Delta E}}{E} \times 100\% \simeq \frac{200}{A}\%}
  \end{gather*}
\end{sol}
\question \\
The 2,200 meters-per-second flux in an ordinary water reactor is 
$1.5 \times 10^{13}$ neutrons/cm$^{2}\!\cdot$sec. 
At what rate are the thermal neutrons absorbed by the water?

\begin{sol}
  \begin{gather*}
    \text{Since water is considered a 1/v nuclei, the absorption density can be found with } F_{a} = \textstyle \sum_{a}(E_{0})\varPhi_{0}\\
    F_{a} = 0.02220 \times 1.5 \times 10^{13} = 3.33\times 10^{11}\, Neutrons/cm^{3}\cdot s
  \end{gather*}
\end{sol}

\question\\
A tiny beryllium target located at the center of a three-dimensional Cartesian 
coordinate system is bombarded by six beams of 0.0253-eV neutrons of intensity 
$3 \times 10^{8}$ neutrons/cm$^{2}\!\cdot$sec, each incident along a different axis.

\begin{enumerate}[label=(\alph*)]
\item What is the 2,200 meters-per-second flux at the target?
\item How many neutrons are absorbed in the target per cm$^{3}$/sec?
\end{enumerate}

\begin{sol}
  \begin{itemize}
    \item{(a)}
    \begin{gather*}
    \text{Thermal neutrons } \approx 0.0253\, eV \approx 2200\, m/s \\
    \varPhi =  6 \times 3 \times 10^{8}\ = 1.8 \times 10^{9} Neutrons/cm^{2}\cdot sec 
    \end{gather*}
    \item{(b)}
    \begin{gather*}
    \text{From table II.3, } \textstyle \sum_{a} = 0.001137\, cm^{-1}\\
    F_{a} = \textstyle \sum_{a}(E_{0})[\varPhi_{1} + \varPhi_{2} + \varPhi_{3} \dots]
    = 0.001137 \times 1.8 \times 10^{6} = 2.05 \times 10^{6} Neutrons/cm^{2}\cdot sec 
    \end{gather*}
  \end{itemize}
\end{sol}

\question\\
The control rods for a certain reactor are made of an alloy of cadmium 
(5 w/o), indium (15 w/o) and silver (80 w/o). Calculate the rate at which 
thermal neutrons are absorbed per gram of this material at a temperature of 
$400\,^{\circ}\mathrm{C}$ in a 2,200 meters-per-second flux of 
$5 \times 10^{13}$ neutrons/cm$^{2}\!\cdot$sec. 

\textit{[Note: Silver is a 1/$\nu$ absorber.]}

\begin{sol}
  \begin{gather*}
  F_{a,t} = \varPhi_{0} \sum_{i}((\textstyle \sum_{a}(E_{0}))_{i}(g_{a}(T))_{i}w_{i}
  \quad \text{with} \quad g_{a}(T) = 1 \quad \text{for 1/v materials}\\
  F_{a,t} = \varPhi_{0}[(\textstyle \sum_{a}(E_{0}))_{Cd}(g_{a}(T))_{Cd}w_{Cd} + (\textstyle \sum_{a}(E_{0}))_{In}(g_{a}(T))_{In}w_{In} + (\textstyle \sum_{a}(E_{0}))_{Ag}(g_{a}(T))_{Ag}w_{Ag}] \\
  F_{a,t} = 5 \times 10^{13}[(113.56)(2.5589)(0.05) + (7.419)(1.1011)(0.15) + (3.725)(1)(0.8)]\\
  F_{a,t} = 9.367\times 10^{14}\, Neutron/cm^{3}.s
  \end{gather*}
\end{sol}

\newpage

\question Find roots of $x^2- 8x = 9$.

\begin{sol}
  We proceed by factoring,
  \begin{align*}
    x^2- 8x - 9     & = 9-9         &  & \text{Subtract 9 on both sides.}         \\
    x^2- x + 9x - 9 & = 0           &  & \text{Breaking the middle term.}         \\
    (x - 1)(x + 9)  & = 0           &  & \text{Pulling out common } (x - 1).      \\
    x               & \in \{1, -9\} &  & f(x)g(x) = 0 \Ra f(x) = 0 \vee g(x) = 0.
  \end{align*}% the empty line bellow is needed to avoid extra space.
  
\end{sol}

\question Figure \ref{wheel} shows two cipher wheels. The left one is from Jeffrey Hoffstein, et al. \cite{hoffstein2008introduction} (pg. 3). Write a Python 3 program that uses it to encrypt: \texttt{FOUR SCORE AND SEVEN YEARS AGO}.

\begin{sol}
  \img<wheel>[0.26]{Cipher wheels.}{cipher.png, diagram.jpg}

  The Python program is given in listing \ref{cpr} and the encryption is given in table \ref{enc}.

  \lstinputlisting[language=Python, caption={Python 3 implementing figure \ref{wheel} left wheel.}, label=cpr]{code/prog.py}

  \tbl<enc>{Caesar cipher} {
    Plain Text  & FOUR & SCORE & AND & SEVEN & YEARS & AGO \\
    Cipher Text & KTZW & XHTWJ & FSI & XJAJS & DJFWX & FLT \\
  }
\end{sol}

% citations
\bibliographystyle{plain}
\bibliography{citations}

\end{document}
